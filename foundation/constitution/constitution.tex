\documentclass[a4paper]{article}
\usepackage[hidelinks]{hyperref}
\usepackage{fontspec,xltxtra,xunicode}
\usepackage[a4paper,top=0.9in,bottom=0.7in,left=1.3in,right=1.3in,includehead,includefoot]{geometry}

\setmainfont{Montserrat}

\usepackage{chngcntr}
\counterwithout{subsection}{section}

\renewcommand*{\theenumi}{\thesubsection.\arabic{enumi}}
\renewcommand*{\thesection}{}

\let\emph\relax
\DeclareTextFontCommand{\emph}{\bfseries}

\newcommand{\generalmeetingthreshold}{35\%}
\newcommand{\companyname}{Cross-Cultural Exchange and Communities Foundation Limited}
\newcommand{\businessname}{Couchers.org Foundation}

\begin{document}
\title{Constitution of the \companyname{}}
\author{A company limited by guarantee
\\ \\
Operating as \businessname{}
\\ \\
ACN: 643 340 309\\
ABN: 20 643 340 309}
\date{}
\maketitle

\newpage

\tableofcontents

\newpage


\section{Preliminary}

\subsection{Name of the company}
\label{subsection-name-of-the-company}

\begin{enumerate}
\item The name of the \emph{company} is \companyname{} (the \emph{company}).
\end{enumerate}

\subsection{Type of company}

\begin{enumerate}
\item The \emph{company} is a not-for-profit public company limited by guarantee which is established to be, and to continue as, a charity.
\end{enumerate}

\subsection{Limited liability of members}

\begin{enumerate}
\item The liability of members is limited to the amount of the guarantee in clause \ref{subsection-guarantee}.
\end{enumerate}

\subsection{The guarantee}
\label{subsection-guarantee}

\begin{enumerate}
\item Each member must contribute an amount not more than \$ 10 (the guarantee) to the property of the \emph{company} if the \emph{company} is wound up while the member is a member, or within 12 months after they stop being a member, and this contribution is required to pay for the:
    \begin{enumerate}
    \item debts and liabilities of the \emph{company} incurred before the member stopped being a member, or
    \item costs of winding up.
    \end{enumerate}
\end{enumerate}

\subsection{Definitions}

\begin{enumerate}
\item In this constitution, words and phrases have the meaning set out in clauses \ref{subsection-definitions} and \ref{subsection-interpretation}.
\end{enumerate}


\section{Charitable purposes and powers}

\subsection{Object}
\label{subsection-object}

\begin{enumerate}
\item The \emph{company} is established to be a charity whose purpose is to advance culture by:
    \begin{enumerate}
    \item designing, building, and maintaining digital tools and activities to facilitate cross-cultural exchange, co-operation, and understanding,
    \item establishing, operating, coordinating, and supporting the communities and individuals that contribute to these digital tools and activities,
    \item instructing and empowering community leaders in building the aforementioned communities,
    \item facilitating events in order to promote awareness of and appreciation of cross-cultural exchange, and
    \item fostering public awareness of and interest in cross-cultural experiences.
    \end{enumerate}
\end{enumerate}

\subsection{Powers}

\begin{enumerate}
\item Subject to clause \ref{subsection-not-for-profit}, the \emph{company} has the following powers, which may only be used to carry out its purposes set out in clause \ref{subsection-object}:
    \begin{enumerate}
    \item the powers of an individual, and
    \item all the powers of a \emph{company} limited by guarantee under the \emph{Corporations Act}.
    \end{enumerate}
\end{enumerate}

\subsection{Not-for-profit}
\label{subsection-not-for-profit}

\begin{enumerate}
\item The \emph{company} must not distribute any income or assets directly or indirectly to its members, except as provided in clauses \ref{clause-good-faith-exception} and \ref{subsection-distribution-of-surplus-assets}.\label{clause-must-not-distribute-to-members}
\item Clause \ref{clause-must-not-distribute-to-members} does not stop the \emph{company} from doing the following things, provided they are done in good faith:\label{clause-good-faith-exception}
    \begin{enumerate}
    \item paying a member for goods or services they have provided or expenses they have properly incurred at fair and reasonable rates or rates more favourable to the \emph{company}, or
    \item making a payment to a member in carrying out the \emph{company}'s charitable purposes.
    \end{enumerate}
\end{enumerate}

\subsection{Amending the constitution}

\begin{enumerate}
\item Subject to clause \ref{clause-remain-charity}, the members may amend this constitution by passing a \emph{special resolution}.
\item The members must not pass a \emph{special resolution} that amends this constitution if passing it causes the \emph{company} to no longer be a charity.\label{clause-remain-charity}
\end{enumerate}


\section{Members}

\subsection{Membership and register of members}

\begin{enumerate}
\item The members of the \emph{company} are:
    \begin{enumerate}
    \item \emph{initial members}, and
    \item any other person that the directors allow to be a member, in accordance with this constitution.
    \end{enumerate}
\item The \emph{company} must establish and maintain a register of members. The register of members must be kept by the secretary and must contain:
    \begin{enumerate}
    \item for each current member:
        \begin{enumerate}
        \item name
        \item address
        \item any alternative address nominated by the member for the service of notices, and
        \item date the member was entered on to the register.
        \end{enumerate}
    \item for each person who stopped being a member in the last 7 years:
        \begin{enumerate}
        \item name
        \item address
        \item any alternative address nominated by the member for the service of notices, and
        \item dates the membership started and ended.
        \end{enumerate}
    \end{enumerate}
\item The \emph{company} must give current members access to the register of members.
\item Information that is accessed from the register of members must only be used in a manner relevant to the interests or rights of members.
\end{enumerate}

\subsection{Who can be a member}

\begin{enumerate}
\item A person who supports the purposes of the \emph{company} is eligible to apply to be a member of the \emph{company} under clause \ref{subsection-how-to-apply-to-become-a-member}.
\item In this clause, `person' means an individual or incorporated body.\label{clause-meaning-of-person}
\end{enumerate}

\subsection{How to apply to become a member}
\label{subsection-how-to-apply-to-become-a-member}

\begin{enumerate}
\item A person (as defined in clause \ref{clause-meaning-of-person}) may apply to become a member of the \emph{company} by writing to the secretary stating that they:
    \begin{enumerate}
    \item want to become a member\label{clause-want-to-become-member}
    \item support the purposes of the \emph{company}, and\label{clause-support-purposes}
    \item agree to comply with the \emph{company}'s constitution, including paying the guarantee under clause \ref{subsection-guarantee} if required.\label{clause-agree-to-constitution}
    \end{enumerate}
\end{enumerate}

\subsection{Directors decide whether to approve membership}

\begin{enumerate}
\item The directors must consider an application for membership within a reasonable time after the secretary receives the application.
\item If the directors approve an application, the secretary must as soon as possible:
    \begin{enumerate}
    \item enter the new member on the register of members, and
    \item write to the applicant to tell them that their application was approved, and the date that their membership started (see clause \ref{subsection-when-a-person-becomes-a-member}).
    \end{enumerate}
\item If the directors reject an application, the secretary must write to the applicant as soon as possible to tell them that their application has been rejected, but does not have to give reasons.
\item For the avoidance of doubt, the directors may approve an application even if the application does not state the matters listed in clauses \ref{clause-want-to-become-member}, \ref{clause-support-purposes} or \ref{clause-agree-to-constitution}. In that case, by applying to be a member, the applicant agrees to those three matters.
\end{enumerate}

\subsection{When a person becomes a member}
\label{subsection-when-a-person-becomes-a-member}

\begin{enumerate}
\item Other than \emph{initial members}, an applicant will become a member when they are entered on the register of members.
\end{enumerate}

\subsection{When a person stops being a member}

\begin{enumerate}
\item A person immediately stops being a member if they:
    \begin{enumerate}
    \item die
    \item are wound up or otherwise dissolved or deregistered (for an incorporated member)
    \item resign, by writing to the secretary
    \item are expelled under clause \ref{subsection-disciplining-members}, or
    \item have not responded within three months to a written request from the secretary that they confirm in writing that they want to remain a member.
    \end{enumerate}
\end{enumerate}


\section{Dispute resolution and disciplinary procedures}

\subsection{Dispute resolution}

\begin{enumerate}
\item The dispute resolution procedure in this clause applies to disputes (disagreements) under this constitution between a member or director and:
    \begin{enumerate}
    \item one or more members
    \item one or more directors, or
    \item the \emph{company}.
    \end{enumerate}
\item A member must not start a dispute resolution procedure in relation to a matter which is the subject of a disciplinary procedure under clause \ref{subsection-disciplining-members} until the disciplinary procedure is completed.
\item Those involved in the dispute must try to resolve it between themselves within 14 days of knowing about it.\label{clause-must-try-to-resolve-between-themselves}
\item If those involved in the dispute do not resolve it under clause \ref{clause-must-try-to-resolve-between-themselves}, they must within 10 days:
    \begin{enumerate}
    \item tell the directors about the dispute in writing
    \item agree or request that a mediator be appointed, and
    \item attempt in good faith to settle the dispute by mediation.
    \end{enumerate}
\item The mediator must:
    \begin{enumerate}
    \item be chosen by agreement of those involved, or
    \item where those involved do not agree:
        \begin{enumerate}
        \item for disputes between members, a person chosen by the directors, or\label{clause-mediator-chose-by-directors}
        \item for other disputes, a person chosen by either the Commissioner of the Australian Charities and Not-for-profits Commission or the president of the law institute or society in the state or territory in which the \emph{company} has its registered office.
        \end{enumerate}
    \end{enumerate}
\item A mediator chosen by the directors under clause \ref{clause-mediator-chose-by-directors}:
    \begin{enumerate}
    \item may be a member or former member of the \emph{company}
    \item must not have a personal interest in the dispute, and
    \item must not be biased towards or against anyone involved in the dispute.
    \end{enumerate}
\item When conducting the mediation, the mediator must:
    \begin{enumerate}
    \item allow those involved a reasonable chance to be heard
    \item allow those involved a reasonable chance to review any written statements
    \item ensure that those involved are given natural justice, and
    \item not make a decision on the dispute.
    \end{enumerate}
\end{enumerate}

\subsection{Disciplining members}
\label{subsection-disciplining-members}

\begin{enumerate}
\item In accordance with this clause, the directors may resolve to warn, suspend or expel a member from the \emph{company} if the directors consider that:\label{clause-discipline-constitution-breach-or-harm}
    \begin{enumerate}
    \item the member has breached this constitution, or
    \item the member's behaviour is causing, has caused, or is likely to cause harm to the \emph{company}.
    \end{enumerate}
\item At least 14 days before the directors' meeting at which a resolution under clause \ref{clause-discipline-constitution-breach-or-harm} will be considered, the secretary must notify the member in writing:
    \begin{enumerate}
    \item that the directors are considering a resolution to warn, suspend or expel the member
    \item that this resolution will be considered at a directors' meeting and the date of that meeting
    \item what the member is said to have done or not done
    \item the nature of the resolution that has been proposed, and
    \item that the member may provide an explanation to the directors, and details of how to do so.
    \end{enumerate}
\item Before the directors pass any resolution under clause \ref{clause-discipline-constitution-breach-or-harm}, the member must be given a chance to explain or defend themselves by:\label{clause-member-may-defend-themselves}
    \begin{enumerate}
    \item sending the directors a written explanation before that directors' meeting, and/or
    \item speaking at the meeting.
    \end{enumerate}
\item After considering any explanation under clause \ref{clause-member-may-defend-themselves}, the directors may:\label{clause-director-discipline-actions}
    \begin{enumerate}
    \item take no further action
    \item warn the member
    \item suspend the member's rights as a member for a period of no more than 12 months
    \item expel the member
    \item refer the decision to an unbiased, independent person on conditions that the directors consider appropriate (however, the person can only make a decision that the directors could have made under this clause), or
    \item require the matter to be determined at a \emph{general meeting}.
    \end{enumerate}
\item The directors cannot fine a member.
\item The secretary must give written notice to the member of the decision under clause \ref{clause-director-discipline-actions} as soon as possible.
\item Disciplinary procedures must be completed as soon as reasonably practical.
\item There will be no liability for any loss or injury suffered by the member as a result of any decision made in good faith under this clause.
\end{enumerate}


\section{General meetings of members}

\subsection{General meetings called by directors}

\begin{enumerate}
\item The directors may call a \emph{general meeting}.
\item If members with at least \generalmeetingthreshold{} of the votes that may be cast at a \emph{general meeting} make a written request to the \emph{company} for a \emph{general meeting} to be held, the directors must:\label{clause-member-general-meeting-vote-threshold}
    \begin{enumerate}
    \item within 21 days of the members' request, give all members notice of a \emph{general meeting}, and
    \item hold the \emph{general meeting} within 2 months of the members' request.
    \end{enumerate}
\item The percentage of votes that members have (in clause \ref{clause-member-general-meeting-vote-threshold}) is to be worked out as at midnight before the members request the meeting.
\item The members who make the request for a \emph{general meeting} must:
    \begin{enumerate}
    \item state in the request any resolution to be proposed at the meeting
    \item sign the request, and
    \item give the request to the \emph{company}.
    \end{enumerate}
\item Separate copies of a document setting out the request may be signed by members if the wording of the request is the same in each copy.
\end{enumerate}

\subsection{General meetings called by members}

\begin{enumerate}
\item If the directors do not call the meeting within 21 days of being requested under clause \ref{clause-member-general-meeting-vote-threshold}, 50\% or more of the members who made the request may call and arrange to hold a \emph{general meeting}.\label{clause-members-may-call-general-meeting}
\item To call and hold a meeting under clause \ref{clause-members-may-call-general-meeting} the members must:
    \begin{enumerate}
    \item as far as possible, follow the procedures for \emph{general meeting}s set out in this constitution
    \item call the meeting using the list of members on the \emph{company}'s member register, which the \emph{company} must provide to the members making the request at no cost, and
    \item hold the \emph{general meeting} within three months after the request was given to the \emph{company}.
    \end{enumerate}
\item The \emph{company} must pay the members who request the \emph{general meeting} any reasonable expenses they incur because the directors did not call and hold the meeting.
\end{enumerate}

\subsection{Annual general meeting}

\begin{enumerate}
\item A \emph{general meeting}, called the annual \emph{general meeting}, must be held:\label{clause-annual-general-meeting-must-be-held}
    \begin{enumerate}
    \item within 18 months after registration of the \emph{company}, and
    \item after the first annual \emph{general meeting}, at least once in every calendar year.
    \end{enumerate}
\item Even if these items are not set out in the notice of meeting, the business of an annual \emph{general meeting} may include:
    \begin{enumerate}
    \item a review of the \emph{company}'s activities
    \item a review of the \emph{company}'s finances
    \item any auditor's report
    \item the election of directors, and
    \item the appointment and payment of auditors, if any.
    \end{enumerate}
\item Before or at the annual \emph{general meeting}, the directors must give information to the members on the \emph{company}'s activities and finances during the period since the last annual \emph{general meeting}.
\item The chairperson of the annual \emph{general meeting} must give members as a whole a reasonable opportunity at the meeting to ask questions or make comments about the management of the \emph{company}.
\end{enumerate}

\subsection{Notice of general meetings}

\begin{enumerate}
\item Notice of a \emph{general meeting} must be given to:
    \begin{enumerate}
    \item each member entitled to vote at the meeting
    \item each director, and
    \item the auditor (if any).
    \end{enumerate}
\item Notice of a \emph{general meeting} must be provided in writing at least 21 days before the meeting.
\item Subject to clause \ref{clause-notice-period-important-resolution}, notice of a meeting may be provided less than 21 days before the meeting if:
    \begin{enumerate}
    \item for an annual \emph{general meeting}, all the members entitled to attend and vote at the annual \emph{general meeting} agree beforehand, or
    \item for any other \emph{general meeting}, members with at least 95\% of the votes that may be cast at the meeting agree beforehand.
    \end{enumerate}
\item Notice of a meeting cannot be provided less than 21 days before the meeting if a resolution will be moved to:\label{clause-notice-period-important-resolution}
    \begin{enumerate}
    \item remove a director
    \item appoint a director in order to replace a director who was removed, or
    \item remove an auditor.
    \end{enumerate}
\item Notice of a \emph{general meeting} must include:
    \begin{enumerate}
    \item the place, date and time for the meeting (and if the meeting is to be held in two or more places, the technology that will be used to facilitate this)
    \item the general nature of the meeting's business
    \item if applicable, that a \emph{special resolution} is to be proposed and the words of the proposed resolution\label{clause-special-resolution}
    \item a statement that members have the right to appoint proxies and that, if a member appoints a proxy:\label{clause-notice-general-meeting-right-to-proxy}
        \begin{enumerate}
        \item the proxy does not need to be a member of the \emph{company}
        \item the proxy form must be delivered to the \emph{company} at its registered address or the address (including an electronic address) specified in the notice of the meeting, and
        \item the proxy form must be delivered to the \emph{company} at least 48 hours before the meeting.
        \end{enumerate}
    \end{enumerate}
\item If a \emph{general meeting} is adjourned (put off) for one month or more, the members must be given new notice of the resumed meeting.
\end{enumerate}

\subsection{Quorum at general meetings}

\begin{enumerate}
\item For a \emph{general meeting} to be held, at least 2 members (a quorum) must be present (in person, by proxy or by representative) for the whole meeting. When determining whether a quorum is present, a person may only be counted once (even if that person is a representative or proxy of more than one member).
\item No business may be conducted at a \emph{general meeting} if a quorum is not present.
\item If there is no quorum present within 30 minutes after the starting time stated in the notice of \emph{general meeting}, the \emph{general meeting} is adjourned to the date, time and place that the chairperson specifies. If the chairperson does not specify one or more of those things, the meeting is adjourned to:
    \begin{enumerate}
    \item if the date is not specified \textendash{} the same day in the next week
    \item if the time is not specified \textendash{} the same time, and
    \item if the place is not specified \textendash{} the same place.
    \end{enumerate}
\item If no quorum is present at the resumed meeting within 30 minutes after the starting time set for that meeting, the meeting is cancelled.
\end{enumerate}

\subsection{Auditor's right to attend meetings}

\begin{enumerate}
\item The auditor (if any) is entitled to attend any \emph{general meeting} and to be heard by the members on any part of the business of the meeting that concerns the auditor in the capacity of auditor.
\item The \emph{company} must give the auditor (if any) any communications relating to the \emph{general meeting} that a member of the \emph{company} is entitled to receive.
\end{enumerate}

\subsection{Representatives of members}
\label{subsection-representatives-of-members}

\begin{enumerate}
\item An incorporated member may appoint as a representative:
    \begin{enumerate}
    \item one individual to represent the member at meetings and to sign circular resolutions under clause \ref{subsection-circular-resolutions-of-members}, and
    \item the same individual or another individual for the purpose of being appointed or elected as a director.
    \end{enumerate}
\item The appointment of a representative by a member must:
    \begin{enumerate}
    \item be in writing
    \item include the name of the representative
    \item be signed on behalf of the member, and
    \item be given to the \emph{company} or, for representation at a meeting, be given to the chairperson before the meeting starts.
    \end{enumerate}
\item A representative has all the rights of a member relevant to the purposes of the appointment as a representative.
\item The appointment may be standing (ongoing).
\end{enumerate}

\subsection{Using technology to hold meetings}

\begin{enumerate}
\item The \emph{company} may hold a \emph{general meeting} at two or more venues using any technology that gives the members as a whole a reasonable opportunity to participate, including to hear and be heard.
\item Anyone using this technology is taken to be present in person at the meeting.
\end{enumerate}

\subsection{Chairperson for general meetings}

\begin{enumerate}
\item The \emph{elected chairperson} is entitled to chair \emph{general meetings}.
\item The members present and entitled to vote at a \emph{general meeting} may choose a director or member to be the chairperson for that meeting if:\label{clause-members-may-choose-chairperson}
    \begin{enumerate}
    \item there is no \emph{elected chairperson}, or
    \item the \emph{elected chairperson} is not present within 30 minutes after the starting time set for the meeting, or
    \item the \emph{elected chairperson} is present but says they do not wish to act as chairperson of the meeting.
    \end{enumerate}
\end{enumerate}

\subsection{Role of the chairperson}

\begin{enumerate}
\item The chairperson is responsible for the conduct of the \emph{general meeting}, and for this purpose must give members a reasonable opportunity to make comments and ask questions (including to the auditor (if any)).
\item The chairperson does not have a casting vote.
\end{enumerate}

\subsection{Adjournment of meetings}

\begin{enumerate}
\item If a quorum is present, a \emph{general meeting} must be adjourned if a majority of \emph{members present} direct the chairperson to adjourn it.
\item Only unfinished business may be dealt with at a meeting resumed after an adjournment.
\end{enumerate}


\section{Members' resolutions and statements}

\subsection{Members' resolutions and statements}
\label{subsection-members-resolutions-and-statements}

\begin{enumerate}
\item Members with at least \generalmeetingthreshold{} of the votes that may be cast on a resolution may give:\label{clause-members-may-propose-and-advertise-resolutions}
    \begin{enumerate}
    \item written notice to the \emph{company} of a resolution they propose to move at a \emph{general meeting} (members' resolution), and/or\label{clause-members-may-propose-resolutions}
    \item a written request to the \emph{company} that the \emph{company} give all of its members a statement about a proposed resolution or any other matter that may properly be considered at a \emph{general meeting} (members' statement).
    \end{enumerate}
\item A notice of a members' resolution must set out the wording of the proposed resolution and be signed by the members proposing the resolution.
\item A request to distribute a members' statement must set out the statement to be distributed and be signed by the members making the request.
\item Separate copies of a document setting out the notice or request may be signed by members if the wording is the same in each copy.
\item The percentage of votes that members have (as described in clause \ref{clause-members-may-propose-and-advertise-resolutions}) is to be worked out as at midnight before the request or notice is given to the \emph{company}.
\item If the \emph{company} has been given notice of a members' resolution under clause \ref{clause-members-may-propose-resolutions}, the resolution must be considered at the next \emph{general meeting} held more than two months after the notice is given.
\item This clause does not limit any other right that a member has to propose a resolution at a \emph{general meeting}.
\end{enumerate}

\subsection{Company must give notice of proposed resolution or distribute statement}
\label{subsection-company-must-give-notice-of-proposed-resolution-or-distribute-statement}

\begin{enumerate}
\item If the \emph{company} has been given a notice or request under clause \ref{subsection-members-resolutions-and-statements}:
    \begin{enumerate}
    \item in time to send the notice of proposed members' resolution or a copy of the members' statement to members with a notice of meeting, it must do so at the \emph{company}'s cost, or
    \item too late to send the notice of proposed members' resolution or a copy of the members' statement to members with a notice of meeting, then the members who proposed the resolution or made the request must pay the expenses reasonably incurred by the \emph{company} in giving members notice of the proposed members' resolution or a copy of the members' statement. However, at a \emph{general meeting}, the members may pass a resolution that the \emph{company} will pay these expenses.\label{clause-members-must-pay-resolution-sending-if-too-late}
    \end{enumerate}
\item The \emph{company} does not need to send the notice of proposed members' resolution or a copy of the members' statement to members if:
    \begin{enumerate}
    \item it is more than 1 000 words long
    \item the directors consider it may be defamatory
    \item clause \ref{clause-members-must-pay-resolution-sending-if-too-late} applies, and the members who proposed the resolution or made the request have not paid the \emph{company} enough money to cover the cost of sending the notice of the proposed members' resolution or a copy of the members' statement to members, or
    \item in the case of a proposed members' resolution, the resolution does not relate to a matter that may be properly considered at a \emph{general meeting} or is otherwise not a valid resolution able to be put to the members.
    \end{enumerate}
\end{enumerate}

\subsection{Circular resolutions of members}
\label{subsection-circular-resolutions-of-members}

\begin{enumerate}
\item Subject to clause \ref{clause-circular-resolution-exceptions}, the directors may put a resolution to the members to pass a resolution without a \emph{general meeting} being held (a circular resolution).
\item The directors must notify the auditor (if any) as soon as possible that a circular resolution has or will be put to members, and set out the wording of the resolution.
\item Circular resolutions cannot be used:\label{clause-circular-resolution-exceptions}
    \begin{enumerate}
    \item for a resolution to remove an auditor, appoint a director or remove a director
    \item for passing a \emph{special resolution}, or
    \item where the \emph{Corporations Act} or this constitution requires a meeting to be held.
    \end{enumerate}
\item A circular resolution is passed if all the members entitled to vote on the resolution sign or agree to the circular resolution, in the manner set out in clause \ref{clause-signing-circular-resolutions} or clause \ref{clause-signing-circular-resolutions-by-email}.
\item Members may sign:\label{clause-signing-circular-resolutions}
    \begin{enumerate}
    \item a single document setting out the circular resolution and containing a statement that they agree to the resolution, or
    \item separate copies of that document, as long as the wording is the same in each copy.
    \end{enumerate}
\item The \emph{company} may send a circular resolution by email to members and members may agree by sending a reply email to that effect, including the text of the resolution in their reply.\label{clause-signing-circular-resolutions-by-email}
\end{enumerate}


\section{Voting at general meetings}

\subsection{How many votes a member has}

\begin{enumerate}
\item Each member has one vote.
\end{enumerate}

\subsection{Challenge to member's right to vote}

\begin{enumerate}
\item A member or the chairperson may only challenge a person's right to vote at a \emph{general meeting} at that meeting.\label{clause-vote-must-be-challenged-at-that-meeting}
\item If a challenge is made under clause \ref{clause-vote-must-be-challenged-at-that-meeting}, the chairperson must decide whether or not the person may vote. The chairperson's decision is final.
\end{enumerate}

\subsection{How voting is carried out}

\begin{enumerate}
\item Voting must be conducted and decided by:
    \begin{enumerate}
    \item a show of hands
    \item a vote in writing, or
    \item another method chosen by the chairperson that is fair and reasonable in the circumstances.
    \end{enumerate}
\item Before a vote is taken, the chairperson must state whether any proxy votes have been received and, if so, how the proxy votes will be cast.
\item On a show of hands, the chairperson's decision is conclusive evidence of the result of the vote.
\item The chairperson and the meeting minutes do not need to state the number or proportion of the votes recorded in favour or against on a show of hands.
\end{enumerate}

\subsection{When and how a vote in writing must be held}

\begin{enumerate}
\item A vote in writing may be demanded on any resolution instead of or after a vote by a show of hands by:\label{clause-show-of-hands-may-be-challenged}
    \begin{enumerate}
    \item at least five \emph{members present}
    \item \emph{members present} with at least 5\% of the votes that may be passed on the resolution on the vote in writing (worked out as at the midnight before the vote in writing is demanded), or
    \item the chairperson.
    \end{enumerate}
\item A vote in writing must be taken when and how the chairperson directs, unless clause \ref{clause-vote-in-writing-for-immediate-decisions} applies.
\item A vote in writing must be held immediately if it is demanded under clause \ref{clause-show-of-hands-may-be-challenged}:\label{clause-vote-in-writing-for-immediate-decisions}
    \begin{enumerate}
    \item for the election of a chairperson under clause \ref{clause-members-may-choose-chairperson}, or
    \item to decide whether to adjourn the meeting.
    \end{enumerate}
\item A demand for a vote in writing may be withdrawn.
\end{enumerate}

\subsection{Appointment of proxy}

\begin{enumerate}
\item A member may appoint a proxy to attend and vote at a \emph{general meeting} on their behalf.
\item A proxy does not need to be a member.
\item A proxy appointed to attend and vote for a member has the same rights as the member to:
    \begin{enumerate}
    \item speak at the meeting
    \item vote in a vote in writing (but only to the extent allowed by the appointment), and
    \item join in to demand a vote in writing under clause \ref{clause-show-of-hands-may-be-challenged}.
    \end{enumerate}
\item An appointment of proxy (proxy form) must be signed by the member appointing the proxy and must contain:
    \begin{enumerate}
    \item the member's name and address
    \item the \emph{company}'s name
    \item the proxy's name or the name of the office held by the proxy, and
    \item the meeting(s) at which the appointment may be used.
    \end{enumerate}
\item A proxy appointment may be standing (ongoing).
\item Proxy forms must be received by the \emph{company} at the address stated in the notice under clause \ref{clause-notice-general-meeting-right-to-proxy} or at the \emph{company}'s registered address at least 48 hours before a meeting.\label{clause-proxy-forms-received-early}
\item A proxy does not have the authority to speak and vote for a member at a meeting while the member is at the meeting.
\item Unless the \emph{company} receives written notice before the start or resumption of a \emph{general meeting} at which a proxy votes, a vote cast by the proxy is valid even if, before the proxy votes, the appointing member:
    \begin{enumerate}
    \item dies
    \item is mentally incapacitated
    \item revokes the proxy's appointment, or
    \item revokes the authority of a representative or agent who appointed the proxy.
    \end{enumerate}
\item A proxy appointment may specify the way the proxy must vote on a particular resolution.
\end{enumerate}

\subsection{Voting by proxy}

\begin{enumerate}
\item A proxy is not entitled to vote on a show of hands (but this does not prevent a member appointed as a proxy from voting as a member on a show of hands).
\item When a vote in writing is held, a proxy:
    \begin{enumerate}
    \item does not need to vote, unless the proxy appointment specifies the way they must vote
    \item if the way they must vote is specified on the proxy form, must vote that way, and
    \item if the proxy is also a member or holds more than one proxy, may cast the votes held in different ways.
    \end{enumerate}
\end{enumerate}


\section{Directors}

\subsection{Number of directors}

\begin{enumerate}
\item The \emph{company} must have at least three and no more than nine directors.
\end{enumerate}

\subsection{Election and appointment of directors}

\begin{enumerate}
\item The initial directors are the people who have agreed to act as directors and who are named as proposed directors in the application for registration of the \emph{company}.
\item Apart from the initial directors and directors appointed under clause \ref{clause-directors-may-appoint-directors}, the members may elect a director by a resolution passed in a \emph{general meeting}.
\item Each of the directors must be appointed by a separate resolution, unless:
    \begin{enumerate}
    \item the members present have first passed a resolution that the appointments may be voted on together, and
    \item no votes were cast against that resolution.
    \end{enumerate}
\item A person is eligible for election as a director of the \emph{company} if they:
    \begin{enumerate}
    \item are a member of the \emph{company}, or a representative of a member of the \emph{company} (appointed under clause \ref{subsection-representatives-of-members})
    \item are nominated by two members or representatives of members entitled to vote (unless the person was previously elected as a director at a \emph{general meeting} and has been a director since that meeting)
    \item give the \emph{company} their signed consent to act as a director of the \emph{company}
    \item are not ineligible by clause \ref{clause-directors-resolution-block}, and
    \item are not ineligible to be a director under the \emph{Corporations Act} or the \emph{ACNC Act}.
    \end{enumerate}
\item The directors may appoint a person as a director to fill a casual vacancy or as an additional director if that person:\label{clause-directors-may-appoint-directors}
    \begin{enumerate}
    \item is a member of the \emph{company}, or a representative of a member of the \emph{company} (appointed under clause \ref{subsection-representatives-of-members})
    \item gives the \emph{company} their signed consent to act as a director of the \emph{company}, and
    \item is not ineligible to be a director under the \emph{Corporations Act} or the \emph{ACNC Act}.
    \end{enumerate}
\item If the number of directors is reduced to fewer than three or is less than the number required for a quorum, the continuing directors may act for the purpose of increasing the number of directors to three (or higher if required for a quorum) or calling a \emph{general meeting}, but for no other purpose.
\end{enumerate}

\subsection{Election of chairperson}
\label{subsection-election-of-chairperson}

\begin{enumerate}
\item The directors must elect a director as the \emph{company}'s \emph{elected chairperson}.
\end{enumerate}

\subsection{Term of office}

\begin{enumerate}
\item At each annual \emph{general meeting}:\label{clause-directors-must-retire}
    \begin{enumerate}
    \item any director appointed by the directors to fill a casual vacancy or as an additional director must retire, and
    \item at least one-third of the remaining directors must retire.\label{clause-third-retire}
    \end{enumerate}
\item The directors who must retire at each annual \emph{general meeting} under clause \ref{clause-third-retire} will be the directors who have been longest in office since last being elected. Where directors were elected on the same day, the director(s) to retire will be decided by lot unless they agree otherwise.
\item Other than a director appointed under clause \ref{clause-directors-may-appoint-directors}, a director's term of office starts at the end of the annual \emph{general meeting} at which they are elected and ends at the end of the annual \emph{general meeting} at which they retire.
\item Each director must retire at least once every three years.
\item A director who retires under clause \ref{clause-directors-must-retire} may nominate for election or re-election, subject to clause \ref{clause-nine-years-special-resolution}.
\item A director who has held office for a continuous period of nine years or more may only be re-appointed or re-elected by a \emph{special resolution}.\label{clause-nine-years-special-resolution}
\end{enumerate}

\subsection{When a director stops being a director}

\begin{enumerate}
\item A director stops being a director if they:
    \begin{enumerate}
    \item give written notice of resignation as a director to the \emph{company}
    \item die
    \item are removed as a director by a resolution of the members
    \item stop being a member of the \emph{company}
    \item are a representative of a member, and that member stops being a member
    \item are a representative of a member, and the member notifies the \emph{company} that the representative is no longer a representative
    \item are absent for 3 consecutive directors' meetings without approval from the directors, or
    \item become ineligible to be a director of the \emph{company} under the \emph{Corporations Act} or the \emph{ACNC Act}.
    \end{enumerate}
\end{enumerate}


\section{Powers of directors}

\subsection{Powers of directors}

\begin{enumerate}
\item The directors are responsible for managing and directing the activities of the \emph{company} to achieve the purposes set out in clause \ref{subsection-object}.
\item The directors may use all the powers of the \emph{company} except for powers that, under the \emph{Corporations Act} or this constitution, may only be used by members.
\item The directors must decide on the responsible financial management of the \emph{company} including:
    \begin{enumerate}
    \item any suitable written delegations of power under clause \ref{subsection-delegation-of-directors-powers}, and
    \item how money will be managed, such as how electronic transfers, negotiable instruments or cheques must be authorised and signed or otherwise approved.
    \end{enumerate}
\item The directors cannot remove a director or auditor. Directors and auditors may only be removed by a members' resolution at a \emph{general meeting}.
\item The directors may pass a directors' resolution to make a member ineligible for election as a director.\label{clause-directors-resolution-block}
\end{enumerate}

\subsection{Delegation of directors' powers}
\label{subsection-delegation-of-directors-powers}

\begin{enumerate}
\item The directors may delegate any of their powers and functions to a committee, a director, an employee of the \emph{company} (such as a chief executive officer) or any other person, as they consider appropriate.
\item The delegation must be recorded in the \emph{company}'s minute book.
\end{enumerate}

\subsection{Payments to directors}

\begin{enumerate}
\item The \emph{company} must not pay fees to a director for acting as a director.
\item The \emph{company} may:\label{clause-payment-to-director}
    \begin{enumerate}
    \item pay a director for work they do for the \emph{company}, other than as a director, if the amount is no more than a reasonable fee for the work done, or
    \item reimburse a director for expenses properly incurred by the director in connection with the affairs of the \emph{company.}
    \end{enumerate}
\item Any payment made under clause \ref{clause-payment-to-director} must be approved by the directors.
\item The \emph{company} may pay premiums for insurance indemnifying directors, as allowed for by law (including the \emph{Corporations Act}) and this constitution.
\end{enumerate}

\subsection{Execution of documents}

\begin{enumerate}
\item The \emph{company} may execute a document without using a common seal if the document is signed by:
    \begin{enumerate}
    \item two directors of the \emph{company}, or
    \item a director and the secretary.
    \end{enumerate}
\end{enumerate}


\section{Duties of directors}

\subsection{Duties of directors}

\begin{enumerate}
\item The directors must comply with their duties as directors under legislation and common law (judge-made law), and with the duties described in governance standard 5 of the regulations made under the \emph{ACNC Act} which are:
    \begin{enumerate}
    \item to exercise their powers and discharge their duties with the degree of care and diligence that a reasonable individual would exercise if they were a director of the \emph{company}
    \item to act in good faith in the best interests of the \emph{company} and to further the charitable purposes of the \emph{company} set out in clause \ref{subsection-object}
    \item not to misuse their position as a director
    \item not to misuse information they gain in their role as a director
    \item to disclose any perceived or actual material conflicts of interest in the manner set out in clause \ref{subsection-conflicts-of-interest}
    \item to ensure that the financial affairs of the \emph{company} are managed responsibly, and
    \item not to allow the \emph{company} to operate while it is insolvent.
    \end{enumerate}
\end{enumerate}

\subsection{Conflicts of interest}
\label{subsection-conflicts-of-interest}

\begin{enumerate}
\item A director must disclose the nature and extent of any actual or perceived material conflict of interest in a matter that is being considered at a meeting of directors (or that is proposed in a circular resolution):
    \begin{enumerate}
    \item to the other directors, or
    \item if all of the directors have the same conflict of interest, to the members at the next \emph{general meeting}, or at an earlier time if reasonable to do so.
    \end{enumerate}
\item The disclosure of a conflict of interest by a director must be recorded in the minutes of the meeting.
\item Each director who has a material personal interest in a matter that is being considered at a meeting of directors (or that is proposed in a circular resolution) must not, except as provided under clauses \ref{clause-conflict-of-interest-directory-may-vote}:
    \begin{enumerate}
    \item be present at the meeting while the matter is being discussed, or
    \item vote on the matter.
    \end{enumerate}
\item A director may still be present and vote if:\label{clause-conflict-of-interest-directory-may-vote}
    \begin{enumerate}
    \item their interest arises because they are a member of the \emph{company}, and the other members have the same interest
    \item their interest relates to an insurance contract that insures, or would insure, the director against liabilities that the director incurs as a director of the \emph{company} (see clause \ref{subsection-insurance})
    \item their interest relates to a payment by the \emph{company} under clause \ref{subsection-indemnity} (indemnity), or any contract relating to an indemnity that is allowed under the \emph{Corporations Act}
    \item the Australian Securities and Investments Commission (ASIC) makes an order allowing the director to vote on the matter, or
    \item the directors who do not have a material personal interest in the matter pass a resolution that:
        \begin{enumerate}
        \item identifies the director, the nature and extent of the director's interest in the matter and how it relates to the affairs of the \emph{company}, and
        \item says that those directors are satisfied that the interest should not stop the director from voting or being present.
        \end{enumerate}
    \end{enumerate}
\end{enumerate}


\section{Directors' meetings}

\subsection{When the directors meet}

\begin{enumerate}
\item The directors may decide how often, where and when they meet.
\end{enumerate}

\subsection{Calling directors' meetings}

\begin{enumerate}
\item A director may call a directors' meeting by giving reasonable notice to all of the other directors.
\item A director may give notice in writing or by any other means of communication that has previously been agreed to by all of the directors.
\end{enumerate}

\subsection{Chairperson for directors' meetings}

\begin{enumerate}
\item The \emph{elected chairperson} is entitled to chair directors' meetings.
\item The directors at a directors' meeting may choose a director to be the chairperson for that meeting if the \emph{elected chairperson} is:
    \begin{enumerate}
    \item not present within 30 minutes after the starting time set for the meeting, or
    \item present but does not want to act as chairperson of the meeting.
    \end{enumerate}
\end{enumerate}

\subsection{Quorum at directors' meetings}

\begin{enumerate}
\item Unless the directors determine otherwise, the quorum for a directors' meeting is a majority (more than 50\%) of directors.
\item A quorum must be present for the whole directors' meeting.
\end{enumerate}

\subsection{Using technology to hold directors' meetings}

\begin{enumerate}
\item The directors may hold their meetings by using any technology (such as video or teleconferencing) that is agreed to by all of the directors.
\item The directors' agreement may be a standing (ongoing) one.
\item A director may only withdraw their consent within a reasonable period before the meeting.
\end{enumerate}

\subsection{Passing directors' resolutions}

\begin{enumerate}
\item A directors' resolution must be passed by a majority of the votes cast by directors present and entitled to vote on the resolution.
\end{enumerate}

\subsection{Circular resolutions of directors}

\begin{enumerate}
\item The directors may pass a circular resolution without a directors' meeting being held.
\item A circular resolution is passed if all the directors entitled to vote on the resolution sign or otherwise agree to the resolution in the manner set out in clause \ref{clause-signing-circular-resolutions-of-directors} or clause \ref{clause-signing-circular-resolutions-of-directors-by-email}.
\item Each director may sign:\label{clause-signing-circular-resolutions-of-directors}
    \begin{enumerate}
    \item a single document setting out the resolution and containing a statement that they agree to the resolution, or
    \item separate copies of that document, as long as the wording of the resolution is the same in each copy.
    \end{enumerate}
\item The \emph{company} may send a circular resolution by email to the directors and the directors may agree to the resolution by sending a reply email to that effect, including the text of the resolution in their reply.\label{clause-signing-circular-resolutions-of-directors-by-email}
\item A circular resolution is passed when the last director signs or otherwise agrees to the resolution in the manner set out in clause \ref{clause-signing-circular-resolutions-of-directors} or clause \ref{clause-signing-circular-resolutions-of-directors-by-email}.
\end{enumerate}


\section{Secretary}

\subsection{Appointment and role of secretary}

\begin{enumerate}
\item The \emph{company} must have at least one secretary, who may also be a director.
\item A secretary must be appointed by the directors (after giving the \emph{company} their signed consent to act as secretary of the \emph{company}) and may be removed by the directors.
\item The directors must decide the terms and conditions under which the secretary is appointed, including any remuneration.
\item The role of the secretary includes:
    \begin{enumerate}
    \item maintaining a register of the \emph{company}'s members, and
    \item maintaining the minutes and other records of \emph{general meeting}s (including notices of meetings), directors' meetings and circular resolutions.
    \end{enumerate}
\end{enumerate}


\section{Minutes and records}

\subsection{Minutes and records}

\begin{enumerate}
\item The \emph{company} must, within one month, make and keep the following records:\label{clause-records-the-company-must-keep}
    \begin{enumerate}
    \item minutes of proceedings and resolutions of \emph{general meetings}
    \item minutes of circular resolutions of members
    \item a copy of a notice of each \emph{general meeting}, and
    \item a copy of a members' statement distributed to members under clause \ref{subsection-company-must-give-notice-of-proposed-resolution-or-distribute-statement}.
    \end{enumerate}
\item The \emph{company} must, within one month, make and keep the following records:\label{clause-director-records-the-company-must-keep}
    \begin{enumerate}
    \item minutes of proceedings and resolutions of directors' meetings (including meetings of any committees), and
    \item minutes of circular resolutions of directors.
    \end{enumerate}
\item To allow members to inspect the \emph{company}'s records:
    \begin{enumerate}
    \item the \emph{company} must give a member access to the records set out in clause \ref{clause-records-the-company-must-keep}, and
    \item the directors may authorise a member to inspect other records of the \emph{company}, including records referred to in clause \ref{clause-director-records-the-company-must-keep} and clause \ref{clause-company-must-keep-financial-records}.
    \end{enumerate}
\item The directors must ensure that minutes of a \emph{general meeting} or a directors' meeting are signed within a reasonable time after the meeting by:
    \begin{enumerate}
    \item the chairperson of the meeting, or
    \item the chairperson of the next meeting.
    \end{enumerate}
\item The directors must ensure that minutes of the passing of a circular resolution (of members or directors) are signed by a director within a reasonable time after the resolution is passed.
\end{enumerate}

\subsection{Financial and related records}

\begin{enumerate}
\item The \emph{company} must make and keep written financial records that:\label{clause-company-must-keep-financial-records}
    \begin{enumerate}
    \item correctly record and explain its transactions and financial position and performance, and
    \item enable true and fair financial statements to be prepared and to be audited.
    \end{enumerate}
\item The \emph{company} must also keep written records that correctly record its operations.
\item The \emph{company} must retain its records for at least 7 years.
\item The directors must take reasonable steps to ensure that the \emph{company}'s records are kept safe.
\end{enumerate}


\section{By-laws}

\subsection{By-laws}

\begin{enumerate}
\item The directors may pass a resolution to make by-laws to give effect to this constitution.
\item Members and directors must comply with by-laws as if they were part of this constitution.
\end{enumerate}


\section{Notice}

\subsection{What is notice}

\begin{enumerate}
\item Anything written to or from the \emph{company} under any clause in this constitution is written notice and is subject to clauses \ref{subsection-notice-to-the-company} to \ref{subsection-when-notice-is-taken-to-be-given}, unless specified otherwise.
\item Clauses \ref{subsection-notice-to-the-company} to \ref{subsection-when-notice-is-taken-to-be-given} do not apply to a notice of proxy under clause \ref{clause-proxy-forms-received-early}.
\end{enumerate}

\subsection{Notice to the company}
\label{subsection-notice-to-the-company}

\begin{enumerate}
\item Written notice or any communication under this constitution may be given to the \emph{company,} the directors or the secretary by:
    \begin{enumerate}
    \item delivering it to the \emph{company}'s registered office
    \item posting it to the \emph{company}'s registered office or to another address chosen by the \emph{company} for notice to be provided, or
    \item sending it to an email address or other electronic address notified by the \emph{company} to the members as the \emph{company}'s email address or other electronic address.
    \end{enumerate}
\end{enumerate}

\subsection{Notice to members}

\begin{enumerate}
\item Written notice or any communication under this constitution may be given to a member:
    \begin{enumerate}
    \item in person
    \item by posting it to, or leaving it at the address of the member in the register of members or an alternative address (if any) nominated by the member for service of notices
    \item sending it to the email or other electronic address nominated by the member as an alternative address for service of notices (if any)
    \item if agreed to by the member, by notifying the member at an email or other electronic address nominated by the member, that the notice is available at a specified place or address (including an electronic address).\label{clause-notice-by-email}
    \end{enumerate}
\item If the \emph{company} does not have an address for the member, the \emph{company} is not required to give notice in person.
\end{enumerate}

\subsection{When notice is taken to be given}
\label{subsection-when-notice-is-taken-to-be-given}

\begin{enumerate}
\item A notice:
    \begin{enumerate}
    \item delivered in person, or left at a the recipient's address, is taken to be given on the day it is delivered
    \item sent by post, is taken to be given on the third day after it is posted with the correct payment of postage costs
    \item sent by email or other electronic method, is taken to be given on the business day after it is sent, and
    \item given under clause \ref{clause-notice-by-email} is taken to be given on the business day after the notification that the notice is available is sent.
    \end{enumerate}
\end{enumerate}


\section{Financial year}

\subsection{Company's financial year}


\begin{enumerate}
\item The \emph{company}'s financial year is from 1 July to 30 June, unless the directors pass a resolution to change the financial year.
\end{enumerate}


\section{Indemnity, insurance and access}

\subsection{Indemnity}
\label{subsection-indemnity}

\begin{enumerate}
\item The \emph{company} indemnifies each officer of the \emph{company} out of the assets of the \emph{company}, to the relevant extent, against all losses and liabilities (including costs, expenses and charges) incurred by that person as an officer of the \emph{company}.
\item In this clause, `officer' means a director or secretary and includes a director or secretary after they have ceased to hold that office.
\item In this clause, `to the relevant extent' means:
    \begin{enumerate}
    \item to the extent that the \emph{company} is not precluded by law (including the \emph{Corporations Act}) from doing so, and
    \item for the amount that the officer is not otherwise entitled to be indemnified and is not actually indemnified by another person (including an insurer under an insurance policy).
    \end{enumerate}
\item The indemnity is a continuing obligation and is enforceable by an officer even though that person is no longer an officer of the \emph{company}.
\end{enumerate}

\subsection{Insurance}
\label{subsection-insurance}

\begin{enumerate}
\item To the extent permitted by law (including the \emph{Corporations Act}), and if the directors consider it appropriate, the \emph{company} may pay or agree to pay a premium for a contract insuring a person who is or has been an officer of the \emph{company} against any liability incurred by the person as an officer of the \emph{company}.
\end{enumerate}

\subsection{Directors' access to documents}

\begin{enumerate}
\item A director has a right of access to the financial records of the \emph{company} at all reasonable times.
\item If the directors agree, the \emph{company} must give a director or former director access to:
    \begin{enumerate}
    \item certain documents, including documents provided for or available to the directors, and
    \item any other documents referred to in those documents.
    \end{enumerate}
\end{enumerate}


\section{Winding up}

\subsection{Surplus assets not to be distributed to members}

\begin{enumerate}
\item If the \emph{company} is wound up, any \emph{surplus assets} must not be distributed to a member or a former member of the \emph{company}, unless that member or former member is a charity described in clause \ref{clause-surplus-assets-must-be-distributed-to-charities}.
\end{enumerate}

\subsection{Distribution of surplus assets}\label{subsection-distribution-of-surplus-assets}

\begin{enumerate}
\item Subject to the \emph{Corporations Act} and any other applicable Act, and any court order, any \emph{surplus assets} that remain after the \emph{company} is wound up must be distributed to one or more charities:\label{clause-surplus-assets-must-be-distributed-to-charities}
    \begin{enumerate}
    \item with charitable purposes similar to, or inclusive of, the purposes in clause \ref{subsection-object}, and
    \item which also prohibit the distribution of any \emph{surplus assets} to its members to at least the same extent as the \emph{company}.
    \end{enumerate}
\item The decision as to the charity or charities to be given the \emph{surplus assets} must be made by a \emph{special resolution} of members at or before the time of winding up. If the members do not make this decision, the \emph{company} may apply to the Supreme Court to make this decision.
\end{enumerate}


\section{Definitions and interpretation}

\subsection{Definitions}
\label{subsection-definitions}

\begin{enumerate}
\item In this constitution:
    \begin{enumerate}
    \item \emph{\textit{ACNC Act}} means the \textit{Australian Charities and Not-for-profits Commission Act 2012} (Cth)
    \item \emph{\textit{company}} means the \emph{company} referred to in clause \ref{subsection-name-of-the-company}
    \item \emph{\textit{Corporations Act}} means the \textit{Corporations Act 2001} (Cth)
    \item \emph{\textit{elected chairperson}} means a person elected by the directors to be the \emph{company}'s chairperson under clause \ref{subsection-election-of-chairperson}
    \item \emph{\textit{general meeting}} means a meeting of members and includes the annual \emph{general meeting}, under clause \ref{clause-annual-general-meeting-must-be-held}
    \item \emph{\textit{initial member}} means a person who is named in the application for registration of the \emph{company}, with their consent, as a proposed member of the \emph{company}
    \item \emph{\textit{member present}} means, in connection with a \emph{general meeting}, a \emph{member present} in person, by representative or by proxy at the venue or venues for the meeting
    \item \emph{\textit{registered charity}} means a charity that is registered under the \emph{ACNC Act}
    \item \emph{\textit{special resolution}} means a resolution:
        \begin{enumerate}
        \item of which notice has been given under clause \ref{clause-special-resolution}, and
        \item that has been passed by at least 75\% of the votes cast by \emph{members present} and entitled to vote on the resolution, and
        \end{enumerate}
    \item \emph{\textit{surplus assets}} means any assets of the \emph{company} that remain after paying all debts and other liabilities of the \emph{company}, including the costs of winding up.
    \end{enumerate}
\end{enumerate}

\subsection{Reading this constitution with the Corporations Act}

\begin{enumerate}
\item The replaceable rules set out in the \emph{Corporations Act} do not apply to the \emph{company}.
\item While the \emph{company} is a \emph{registered charity}, the \emph{ACNC Act} and the \emph{Corporations Act} override any clauses in this constitution which are inconsistent with those Acts.
\item If the \emph{company} is not a \emph{registered charity} (even if it remains a charity), the \emph{Corporations Act} overrides any clause in this constitution which is inconsistent with that Act.
\item A word or expression that is defined in the \emph{Corporations Act}, or used in that Act and covering the same subject, has the same meaning as in this constitution.
\end{enumerate}

\subsection{Interpretation}
\label{subsection-interpretation}

\begin{enumerate}
\item In this constitution:
    \begin{enumerate}
    \item the words `including', `for example', or similar expressions mean that there may be more inclusions or examples than those mentioned after that expression, and
    \item reference to an Act includes every amendment, re-enactment, or replacement of that Act and any subordinate legislation made under that Act (such as regulations).
    \end{enumerate}
\end{enumerate}

\end{document}
